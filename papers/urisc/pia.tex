\section{M6809 Monitor program}

This is the short assembly language program used  to support communications between the 68000 monitor and the Ultimate RISC.
Originally a simple repetitive loop to transfer the contents of the PIA to shared memory locations, and to accept update requests from the 68000 via shared
memory locations, it has been upgraded to 

\begin{verbatim}
* Virtual PIA program
* 13/3/89 Steve Loughran
* optimised 20/4/89
* loop mode 9/5/89
* register get/put 11/5/89

* maintains an image of the PIA in addresses COPY to COPY+4
* to write to an address 68K program writes data into location DATA
* and address (as offset from PIA) into addr - with bit 7 =1
* 6809 program will then copy data to the location, 
* update the image of the PIA and then clear ADDR

* PIA address
PIA equ $C200
porta  equ $c200 *PIA port A
portb  equ $c202 *PIA port B
addr   equ    0  *host write address (offset from PIA)
data   equ    1  *byte to write
copy   equ    2  *copy of port A

times  equ    4  *loop count
val1   equ    8  *pattern 1 (usually clock high)
val2   equ    9  *pattern 2 (usually clock bits low)
c1     equ  $10
c2     equ  $11
regs   equ  $1E   *no. of shared registers
dir    equ  $1F   *direction of get/put
SSRs   equ  $20   *Shared SSR image array

* system stack
sstack equ $BEF0

       org $100
start  lds #sstack
       lda #$ff
       sta data
       clr addr
       clr times
       clr a          *always keep a-reg empty
loop1  ldb PIA        *update the image
       stb copy
loop2  
       ldb addr
       bne update      * see if address written to
       ldx times
       beq cont        *see if loop mode requested
       bsr lmode
       bra loop2
cont   ldb dir         *see if get/put requested
       beq loop1
       and#1
       beq get         *test for `get' request
       bsr psub
       bra loop2
get    bsr gsub
       bra loop2

* now copy data to address
update andb #127      *mask hi bit
       tfr d,x        *copy to X index
       ldb data
       stb PIA,x
       ldb PIA
       stb copy      *update the image
       sta addr      *acknowledge
       bra loop2     *continue

*loop mode, loop 'times' times
* storing first val1 then val2 into portb
* with a delay proportional to 'pause' between each write
* terminates with val1 in portB
* for use in generating dclk & clk signals

lmode  lda val1
       ldb val2
lm1    sta portb
       nop
       nop
       nop
       stb portb
       leax -1,x
       bne lm1
       sta portb
       lda porta
       sta copy
       clr a
       clr times
       clr times+1
       rts


*subroutine to get all the registers into the shared memory.
*the SSRS must have already been instructed to copy the 
*registers to their shadows
*the clocking pattern should be set up in val1 & val2

gsub   ldx#0
       ldb val1
       stb portb
gl1    clr SSrs,x
       lda#8
gl2    ldb porta
       rolb
       rolb
       rol SSRs,x
       ldb val2
       stb portb
       nop
       nop
       ldb val1
       stb portb
       deca
       bne gl2
       leax 1,x
       cmpx regs-1
       bne gl1
       lda porta
       sta copy
       clr dir
       clr a
       rts

*the routine to put all the registers bask to the SSRS

psub   ldx#0
       ldb val1
       stb portb
pl1    ldy#8
       lda SSRs,x       
pl2    tfr a,b
       andb#128
       stb porta
       nop
       nop
       ldb val2
       stb portb
       rola
       nop
       nop
       ldb val1
       stb portb
       leay -1,y
       bne pl2
       leax 1,x
       cmpx regs-1
       bne pl1
       lda porta
       sta copy
       clr dir
       clr a
       rts

*reset vector
       org $fffe
       fdb start

end
\end{verbatim}
