\chapter{Using the Monitor Program}

This appendix describes how to to use the monitor program to control the Ultimate RISC.

To make use of the monitor it is best to have read the main body of the project report, paying special attention the the host interface.

One should also consult the CS department report on using the M6809 monitor \cite{cs2:mon}.

\section{Main Points}
This monitor program is derived from the 6809 monitor, as it uses this processor to communicate with the Ultimate RISC.
Some of the basic 6809 commands are still available, although  renamed.

The commands it supports range from high level commands to load and run programs on the Ultimate RISC, down to basic signal manipulation.

While the monitor  examination of the Ultimate RISC's registers, these
are only a copy of the actual registers.
An explicit command to get an updated copy of these registers must be issued whenever the current state is required.

\section{Files}

The monitor consists of a number of files stored in the directory `{\bf c::sal}'.
The main 68000 based program is titled {\bf monitor}.
The 6809 Virtual PIA program is stored in the file {\bf pia5.obj}.
Access to these two programs, and the two files {\bf registers} and {\bf states},
is required to use the monitor.

\section{Using the monitor}

An APM with a 6809 board  and Wyse terminal  is a prerequisite to running this program.
To start using the monitor issue the command
\begin{verbatim}
sal:monitor
\end{verbatim}
After a short pause the monitor screen display will appear.
This comprises a status display at the top of the screen with a command region below .
One status line shows the current state of the Control Unit as a mnemonic name.
The line beneath this gives the state of all the I/O lines; a signal is 'true' when its name appears in inverse video.
This display is updated after every command.

The command {\bf help} will list  all  other available commands.

To stop the monitor type either {\bf quit} or control-Y.

\section{States}
The Control Unit can be in any one of thirty-two states.
Only about half of these are actually used, and the unit should never enter any of the other states.
The most important states are listed below.
\subsubsection{Halted}
This is the power up state, also entered after a memory read is prevented by the memory unit.

\subsubsection{Waiting}
The control unit is ready and waiting for use.
It can be instructed to:-
\begin{itemize}
\item  read and write registers
\item  read and write memory
\item  execute instructions
\end{itemize}

\subsubsection{No Board!}
The monitor can not detect the presence of the Ultimate RISC.
It may not be installed or powered up.

\subsubsection{Memory Access States}
Any state with the name {\bf read-X} or {\bf write-X} is part of the host memory read
or write cycles. 
These should not be encountered during normal use.

\subsubsection{Unknown States}
Any state named {\bf unknown-XX} is not part of the programmed state machine.
If any of these states appear then it is probably due to a metastability problem;
if a known state can not be reached then a power down reset is obligatory.

\subsubsection{Instruction Execution}
Other named states are steps in the process of instruction execution.

\section{Basic Commands}

\subsubsection{Register Manipulation}
The monitor's copy of the registers can be listed with the command {\bf registers}.
This lists each register name with its value in hexadecimal.
To update this copy with the Ultimate RISC's current state, use the command
{\bf get}. 
This lists the registers after updating them.

\subsubsection{Memory Access}
A region of memory can be examined by the command:-
\begin{verbatim}
     read <from> <to>
\end{verbatim}  
This gives a list of the contents of the memory locations between the two addresses.
Any memory location  with a horizontal line in place of a numeric value for the contents is a write only address.

To change an address in memory use the command
\begin{verbatim}
      write <address> <value>
\end{verbatim}

To download a file into memory, using the format listed in this report, issue the command
\begin{verbatim}
      download <filename>
      \end{verbatim}

\subsubsection{Instruction Execution}
Instruction execution can be controlled either by the  host or the on board clock.
To control instructions from the host issue the command
\begin{verbatim}
	execute <count>
\end{verbatim}
The specified number of instructions will be executed, unless a halt state is reached earlier. 
If a zero count is given then instructions will be executed until halted.
A count of the number of instructions executed is provided afterwards.

To let the Ultimate RISC execute instructions at full speed, issue the command
{\bf spin}. 

To stop instruction execution at any time  press control-C.

\section{Low Level Commands}

\subsubsection{Signal Manipulation}
The output signals can be directly toggled by entering the signal name followed by a zero or a one.
This updates the state display, but the changes are not sent to the ultimate RISC until explicitly transmitted with the command {\bf tx}.

\subsubsection{Clock Control}
A single clock pulse can be transmitted to the Ultimate RISC by the command {\bf step}.
To place the Ultimate RISC into freerunning mode  clear the {\bf freerun} signal and transmit the change.

\subsubsection{Low Level  Register Manipulation}
To alter the value of a register, use the command:-
\begin{verbatim}
      set <register> <value>
\end{verbatim}
Only the address and data registers are currently reloaded, although the control unit can be reprogrammed to load the instruction register as well.
This copy must be shifted back to the shadow registers by the command {\bf put}.
The Control Unit must then be instructed to load the address and data registers from
their shadows; the command {\bf store} attempts to do this.


\subsubsection{Memory Testing}
Two commands are provided to test the correct operation of memory.
They both write a pattern of bits into a memory region, then read back result to log any differences.
This is useful in testing for the correct operation of both the control unit and memory.

To test an arbitrary region of memory use the command:-
\begin{verbatim}
     memtest <from> <to> <pattern>
\end{verbatim}
This writes the pattern into all the memory locations, unless the pattern is equal
to the start address.
In this special case the pattern is incremented so it is always equal to the location written to.
This provides an effective test for folded memory locations.

An extended test of RAM is also available:-
\begin{verbatim}
      soak <repetitions>,<failures>
\end{verbatim}
This repeatedly tests RAM with patterns of alternate bits and addresses until
either the specified number of repetitions has been completed or the stated number of failures has been exceeded.
If either of the the two parameters is zero then that parameter is ignored ---either repeating indefinately or ignoring the number of failures.

\subsubsection{6809 commands}
Most of the 6809 commands have been removed, with only four still supported.
They are:-
\begin{description}
	\item[obj] download object code file
	\item[run] execute a program (was {\bf go})
	\item[dump] dump memory region
	\item[byte] write a byte to memory
	\end{description}
If one is experimenting with 6809 programs the command {\bf initialise} is useful;
this attempts to reinitialise the PIA port and common variables.
	
\section{Maintenance}
The IMP source code for the monitor is in the file `{\bf sal:monitor}'.
The 6809 assembly language program is in `{\bf sal:pia5.asm}'.
There are also two files to describe the states and the registers.

\subsection{States}
This file contains thirty-two lines, each listing the mnemonic name for a state.
It should be updated with every reprogramming of the control EPLD.

\subsection{Registers}
This file records the name of each register and their positions with the SSR chain.
Each register IC is only eight bits wide, so a thirty two bit register is built out of four SSRS.
These sub-registers do not have to be adjacent to one another in the SSR chain.

The file starts with a number listing the number of registers in the chain,
followed by the list of sub-registers, one to a line.
Each sub-register is listed with the name of the register and the byte within that register which it is, in the range 0 to 3.
The order of the registers in the file is that of the SSR chain,
with the register connected to the host's SDI input first in this list.

\section{Error Messages}
There are four  messages which the monitor issues, indicating something
is not quite right with the Ultimate RISC.
They may be followed with some technical detail indicating what the monitor was trying to do, and what the response is.

{\samepage
\begin{verbatim}
      `0' not passed through SSR chain
                    or 
      `1' not passed through SSR chain
   \end{verbatim}
These messages state that control signals have not propagated all the way through the chain of registers.
The response to either of these should follow the following sequence:-
\begin{enumerate}
\item try to {\bf get} or {\bf put} the registers a few more times
\item check the board is powered up and plugged in
\item check the number of registers installed matched that stated in the file `Registers'
\item push all the registers firmly into their sockets 
\end{enumerate}
}

\begin{verbatim}
       Not in the right state!
\end{verbatim}
The control unit is not in the correct state for the required operation ---usually involving a memory access.
Try issuing the {\bf step} command a few times to see if the waiting or halted states can be reached.
If the step command has no effect then the board must be reset.

\begin{verbatim}
	Attempted SSR writeback!
\end{verbatim}
This error should never occur in normal use, unless manually manipulating the output lines.
The current control signals, if transmitted, could have caused the SSR registers to try driving their inputs ---a dangerous operation which has been automatically intercepted.
Consult the AMD databook to see why this should be avoided \cite{amd:logic}.











 
