
\chapter{The Formal Specification and Simulation}

This appendix lists the second specification of the computer,
along with the simulation derived from it.
It is composed of a number of files, some of which the specification alone
uses, some of which the simulation makes use of.
The Lambda specification describes the operation of the ALU on a  component by component basis, including timing constraints.
The simulation is executable in ML, and describes the entire computer.
As such it forms another specification, with the functions being described without too much implementation detail.

\section{Lambda Specification}
The lambda specification provides the core of the code for the simulation, but is too powerful to be compiled by ML.
It can be used to some prove properties of the computer.
It can also be used to verify the speed of operation of the ALU.
First  defined are the thirty-two bit and fifteen bit boolean tuples used
in the specification.
Constants and conversion functions are also listed; these conversion functions do not work correctly in SML or New Jersey ML due to overflow conditions.
\begin{verbatim}
(* Datatypes.L v1.8
   ==================
Datatypes used in ultimate RISC: revised version
2/1/89 sal *)
(* Int15 *)
type Int15=(bool * bool * bool * bool * bool * bool * bool *bool
                * bool * bool * bool * bool * bool * bool * bool);
val Zero15=(false,false,false,false,false,false,false,
                false,false,false,false,false,false,false,false):Int15;
val Maxint15=(true,true,true,true,true,true,true,
                true,true,true,true,true,true,true,true):Int15;
(* bit extraction functions *)
fun addressBit14 ((b,_,_,_,_,_,_,_,_,_,_,_,_,_,_):Int15)=b:bool;
fun addressBit13 ((_,b,_,_,_,_,_,_,_,_,_,_,_,_,_):Int15)=b:bool;
fun addressBit12 ((_,_,b,_,_,_,_,_,_,_,_,_,_,_,_):Int15)=b:bool;
fun addressBit11 ((_,_,_,b,_,_,_,_,_,_,_,_,_,_,_):Int15)=b:bool;
fun addressBit10 ((_,_,_,_,b,_,_,_,_,_,_,_,_,_,_):Int15)=b:bool;
fun addressBit9  ((_,_,_,_,_,b,_,_,_,_,_,_,_,_,_):Int15)=b:bool;
fun addressBit8  ((_,_,_,_,_,_,b,_,_,_,_,_,_,_,_):Int15)=b:bool;
fun addressBit7  ((_,_,_,_,_,_,_,b,_,_,_,_,_,_,_):Int15)=b:bool;
fun addressBit6  ((_,_,_,_,_,_,_,_,b,_,_,_,_,_,_):Int15)=b:bool;
fun addressBit5  ((_,_,_,_,_,_,_,_,_,b,_,_,_,_,_):Int15)=b:bool;
fun addressBit4  ((_,_,_,_,_,_,_,_,_,_,b,_,_,_,_):Int15)=b:bool;
fun addressBit3  ((_,_,_,_,_,_,_,_,_,_,_,b,_,_,_):Int15)=b:bool;
fun addressBit2  ((_,_,_,_,_,_,_,_,_,_,_,_,b,_,_):Int15)=b:bool;
fun addressBit1  ((_,_,_,_,_,_,_,_,_,_,_,_,_,b,_):Int15)=b:bool;
fun addressBit0  ((_,_,_,_,_,_,_,_,_,_,_,_,_,_,b):Int15)=b:bool;
(* a little type conversion utility *)
fun booltoNat false=0
 |  booltoNat true=1;
fun Int15toNat (i:Int15)=
        (booltoNat (addressBit0 i)) + 
	2 * (booltoNat (addressBit1 i)) +
	4 * (booltoNat (addressBit2 i))
        + 8*(booltoNat (addressBit3 i)) +
	16 * (booltoNat (addressBit4 i)) +
	32 * (booltoNat (addressBit5 i))
        + 64*(booltoNat (addressBit6 i)) +
	128*(booltoNat (addressBit7 i))
        + 256*(booltoNat (addressBit8 i)) +
	512*(booltoNat (addressBit9 i))
        + 1024*(booltoNat (addressBit10 i)) +
	2048*(booltoNat (addressBit11 i))
        + 4096*(booltoNat (addressBit12 i)) +
	8192*(booltoNat (addressBit13 i))
        + 16384*(booltoNat (addressBit14 i));
(* Int32 -very similar to Int15 *)
type Int32=(bool * bool * bool * bool * bool * bool * bool *bool *
              bool * bool * bool * bool * bool * bool * bool * bool*
              bool * bool * bool * bool * bool * bool * bool *bool*
              bool * bool * bool * bool * bool * bool * bool *bool);
val Zero32=(false,false,false,false,false,false,false,false,
                false,false,false,false,false,false,false,false,
                false,false,false,false,false,false,false,false,
                false,false,false,false,false,false,false,false):Int32;
val Maxint32=(true,true,true,true,true,true,true,true,
                true,true,true,true,true,true,true,true,
                true,true,true,true,true,true,true,true,
                true,true,true,true,true,true,true,true):Int32;
(* bit extraction functions *)
fun dataBit31 ((b,_,_,_,_,_,_,_,_,_,_,_,_,_,_,_,
                 _,_,_,_,_,_,_,_,_,_,_,_,_,_,_,_):Int32)=b;
fun dataBit30 ((_,b,_,_,_,_,_,_,_,_,_,_,_,_,_,_,
                 _,_,_,_,_,_,_,_,_,_,_,_,_,_,_,_):Int32)=b;
fun dataBit29 ((_,_,b,_,_,_,_,_,_,_,_,_,_,_,_,_,
                 _,_,_,_,_,_,_,_,_,_,_,_,_,_,_,_):Int32)=b;
fun dataBit28 ((_,_,_,b,_,_,_,_,_,_,_,_,_,_,_,_,
                 _,_,_,_,_,_,_,_,_,_,_,_,_,_,_,_):Int32)=b;
fun dataBit27 ((_,_,_,_,b,_,_,_,_,_,_,_,_,_,_,_,
                 _,_,_,_,_,_,_,_,_,_,_,_,_,_,_,_):Int32)=b;
fun dataBit26 ((_,_,_,_,_,b,_,_,_,_,_,_,_,_,_,_,
                 _,_,_,_,_,_,_,_,_,_,_,_,_,_,_,_):Int32)=b;
fun dataBit25 ((_,_,_,_,_,_,b,_,_,_,_,_,_,_,_,_,
                 _,_,_,_,_,_,_,_,_,_,_,_,_,_,_,_):Int32)=b;
fun dataBit24 ((_,_,_,_,_,_,_,b,_,_,_,_,_,_,_,_,
                 _,_,_,_,_,_,_,_,_,_,_,_,_,_,_,_):Int32)=b;
fun dataBit23 ((_,_,_,_,_,_,_,_,b,_,_,_,_,_,_,_,
                 _,_,_,_,_,_,_,_,_,_,_,_,_,_,_,_):Int32)=b;
fun dataBit22 ((_,_,_,_,_,_,_,_,_,b,_,_,_,_,_,_,
                 _,_,_,_,_,_,_,_,_,_,_,_,_,_,_,_):Int32)=b;
fun dataBit21 ((_,_,_,_,_,_,_,_,_,_,b,_,_,_,_,_,
                 _,_,_,_,_,_,_,_,_,_,_,_,_,_,_,_):Int32)=b;
fun dataBit20 ((_,_,_,_,_,_,_,_,_,_,_,b,_,_,_,_,
                 _,_,_,_,_,_,_,_,_,_,_,_,_,_,_,_):Int32)=b;
fun dataBit19 ((_,_,_,_,_,_,_,_,_,_,_,_,b,_,_,_,
                 _,_,_,_,_,_,_,_,_,_,_,_,_,_,_,_):Int32)=b;
fun dataBit18 ((_,_,_,_,_,_,_,_,_,_,_,_,_,b,_,_,
                 _,_,_,_,_,_,_,_,_,_,_,_,_,_,_,_):Int32)=b;
fun dataBit17 ((_,_,_,_,_,_,_,_,_,_,_,_,_,_,b,_,
                 _,_,_,_,_,_,_,_,_,_,_,_,_,_,_,_):Int32)=b;
fun dataBit16 ((_,_,_,_,_,_,_,_,_,_,_,_,_,_,_,b,
                 _,_,_,_,_,_,_,_,_,_,_,_,_,_,_,_):Int32)=b;
fun dataBit15 ((_,_,_,_,_,_,_,_,_,_,_,_,_,_,_,_,
                 b,_,_,_,_,_,_,_,_,_,_,_,_,_,_,_):Int32)=b;
fun dataBit14 ((_,_,_,_,_,_,_,_,_,_,_,_,_,_,_,_,
		 _,b,_,_,_,_,_,_,_,_,_,_,_,_,_,_):Int32)=b;
fun dataBit13 ((_,_,_,_,_,_,_,_,_,_,_,_,_,_,_,_,
                 _,_,b,_,_,_,_,_,_,_,_,_,_,_,_,_):Int32)=b;
fun dataBit12 ((_,_,_,_,_,_,_,_,_,_,_,_,_,_,_,_,
                 _,_,_,b,_,_,_,_,_,_,_,_,_,_,_,_):Int32)=b;
fun dataBit11 ((_,_,_,_,_,_,_,_,_,_,_,_,_,_,_,_,
                 _,_,_,_,b,_,_,_,_,_,_,_,_,_,_,_):Int32)=b;
fun dataBit10 ((_,_,_,_,_,_,_,_,_,_,_,_,_,_,_,_,
                 _,_,_,_,_,b,_,_,_,_,_,_,_,_,_,_):Int32)=b;
fun dataBit9  ((_,_,_,_,_,_,_,_,_,_,_,_,_,_,_,_,
                 _,_,_,_,_,_,b,_,_,_,_,_,_,_,_,_):Int32)=b;
fun dataBit8  ((_,_,_,_,_,_,_,_,_,_,_,_,_,_,_,_,
                 _,_,_,_,_,_,_,b,_,_,_,_,_,_,_,_):Int32)=b;
fun dataBit7  ((_,_,_,_,_,_,_,_,_,_,_,_,_,_,_,_,
                 _,_,_,_,_,_,_,_,b,_,_,_,_,_,_,_):Int32)=b;
fun dataBit6  ((_,_,_,_,_,_,_,_,_,_,_,_,_,_,_,_,
                 _,_,_,_,_,_,_,_,_,b,_,_,_,_,_,_):Int32)=b;
fun dataBit5  ((_,_,_,_,_,_,_,_,_,_,_,_,_,_,_,_,
                 _,_,_,_,_,_,_,_,_,_,b,_,_,_,_,_):Int32)=b;
fun dataBit4  ((_,_,_,_,_,_,_,_,_,_,_,_,_,_,_,_,
                 _,_,_,_,_,_,_,_,_,_,_,b,_,_,_,_):Int32)=b;
fun dataBit3  ((_,_,_,_,_,_,_,_,_,_,_,_,_,_,_,_,
                 _,_,_,_,_,_,_,_,_,_,_,_,b,_,_,_):Int32)=b;
fun dataBit2  ((_,_,_,_,_,_,_,_,_,_,_,_,_,_,_,_,
                 _,_,_,_,_,_,_,_,_,_,_,_,_,b,_,_):Int32)=b;
fun dataBit1  ((_,_,_,_,_,_,_,_,_,_,_,_,_,_,_,_,
                 _,_,_,_,_,_,_,_,_,_,_,_,_,_,b,_):Int32)=b;
fun dataBit0  ((_,_,_,_,_,_,_,_,_,_,_,_,_,_,_,_,
                 _,_,_,_,_,_,_,_,_,_,_,_,_,_,_,b):Int32)=b;
(* another little type conversion utility *)
fun Int32toNat i=
        (booltoNat ( dataBit0 i)) + 
	2 * (booltoNat (dataBit1 i)) + 
	4 * (booltoNat (dataBit2 i))
        + 8*(booltoNat (dataBit3 i)) +
	16 * (booltoNat (dataBit4 i)) +
	32 * (booltoNat (dataBit5 i))
        + 64*(booltoNat (dataBit6 i)) 
	+ 128*(booltoNat (dataBit7 i))
        + 256*(booltoNat (dataBit8 i)) 
	+ 512*(booltoNat (dataBit9 i))
        + 1024*(booltoNat (dataBit10 i)) 
	+ 2048*(booltoNat (dataBit11 i))
        + 4096*(booltoNat (dataBit12 i)) 
	+ 8192*(booltoNat (dataBit13 i))
        + 16384*(booltoNat (dataBit14 i)) 
	+ 32768*(booltoNat (dataBit15 i))
        + 65536*(booltoNat (dataBit16 i)) 
	+ 131072*(booltoNat (dataBit17 i))
        + 262144*(booltoNat (dataBit18 i)) 
	+ 524288*(booltoNat (dataBit19 i))
        + 1048576*((booltoNat (dataBit20 i)) 
         + 2*((booltoNat (dataBit21 i))
          + 2*((booltoNat (dataBit22 i))
           + 2*((booltoNat (dataBit23 i))
            + 2*((booltoNat (dataBit24 i))
             + 2*((booltoNat (dataBit25 i))
              + 2*((booltoNat (dataBit26 i))
               + 2*((booltoNat (dataBit27 i))
                + 2*((booltoNat (dataBit28 i))
                 + 2*((booltoNat (dataBit29 i))
                  + 2*((booltoNat (dataBit30 i))
                   + 2*(booltoNat (dataBit31 i))
                  )
                 )
                )
               )
              )
             )
            )
           )
          )
         )
       ) 
      ;

(* definition of 4-tuples and 8-tuples (used in the alu) *)
type 'a four_tuple = ('a * 'a * 'a * 'a);
type nibble=bool four_tuple;
fun one4 (_,_,_,a) = a;
fun two4 (_,_,a,_)=a;
fun three4 (_,a,_,_)=a;
fun four4 (a,_,_,_)=a;
type 'a eight_tuple = ('a * 'a * 'a * 'a * 'a * 'a * 'a * 'a);
(* and a function to convert a 32 bit number into an 8-tuple
of boolean 4-tuples *)
fun split d=
        ((dataBit31 d,dataBit30 d,dataBit29 d,dataBit28 d),
         (dataBit27 d,dataBit26 d,dataBit25 d,dataBit24 d),
         (dataBit23 d,dataBit22 d,dataBit21 d,dataBit20 d),
         (dataBit19 d,dataBit18 d,dataBit17 d,dataBit16 d),
         (dataBit15 d,dataBit14 d,dataBit13 d,dataBit12 d),
         (dataBit11 d,dataBit10 d,dataBit9 d,dataBit8 d),
         (dataBit7 d,dataBit6 d,dataBit5 d,dataBit4 d),
         (dataBit3 d,dataBit2 d,dataBit1 d,dataBit0 d));
type byte=bool eight_tuple;
fun split8 d=
        ((dataBit31 d,dataBit30 d,dataBit29 d,dataBit28 d,
         dataBit27 d,dataBit26 d,dataBit25 d,dataBit24 d),
         (dataBit23 d,dataBit22 d,dataBit21 d,dataBit20 d,
         dataBit19 d,dataBit18 d,dataBit17 d,dataBit16 d),
         (dataBit15 d,dataBit14 d,dataBit13 d,dataBit12 d,
         dataBit11 d,dataBit10 d,dataBit9 d,dataBit8 d),
         (dataBit7 d,dataBit6 d,dataBit5 d,dataBit4 d,
         dataBit3 d,dataBit2 d,dataBit1 d,dataBit0 d));
fun Int15toByte ((a6,a5,a4,a3,a2,a1,a0,b7,b6,b5,b4,b3,b2,b1,b0):Int15)=
        ((false,a6,a5,a4,a3,a2,a1,a0),(b7,b6,b5,b4,b3,b2,b1,b0));
(* inverse functions *)
fun BytetoInt15 (_,a6,a5,a4,a3,a2,a1,a0) (b7,b6,b5,b4,b3,b2,b1,b0)=
        (a6,a5,a4,a3,a2,a1,a0,b7,b6,b5,b4,b3,b2,b1,b0);
fun merge8 (a7,a6,a5,a4,a3,a2,a1,a0) (b7,b6,b5,b4,b3,b2,b1,b0)
           (f7,f6,f5,f4,f3,f2,f1,f0) (g7,g6,g5,g4,g3,g2,g1,g0)=
        ((a7,a6,a5,a4,a3,a2,a1,a0,b7,b6,b5,b4,b3,b2,b1,b0,
         f7,f6,f5,f4,f3,f2,f1,f0,g7,g6,g5,g4,g3,g2,g1,g0):Int32);
fun merge  (a7,a6,a5,a4) (a3,a2,a1,a0) (b7,b6,b5,b4) (b3,b2,b1,b0)
           (f7,f6,f5,f4) (f3,f2,f1,f0) (g7,g6,g5,g4) (g3,g2,g1,g0)=
        ((a7,a6,a5,a4,a3,a2,a1,a0,b7,b6,b5,b4,b3,b2,b1,b0,
         f7,f6,f5,f4,f3,f2,f1,f0,g7,g6,g5,g4,g3,g2,g1,g0):Int32);
fun splitInt15 ((a6,a5,a4,a3,a2,a1,a0,b7,b6,b5,b4,b3,b2,b1,b0):Int15)=
        ((false,a6,a5,a4),(a3,a2,a1,a0),(b7,b6,b5,b4),(b3,b2,b1,b0));
fun mergeInt15 (_,a6,a5,a4) (a3,a2,a1,a0) (b7,b6,b5,b4) (b3,b2,b1,b0)=
        ((a6,a5,a4,a3,a2,a1,a0,b7,b6,b5,b4,b3,b2,b1,b0):Int15);
(* plus one to split & merge into4 tuples of 7 plus 4 leftovers *)
fun merge7 (h3,h2,h1,h0)
	   (a6,a5,a4,a3,a2,a1,a0) (b6,b5,b4,b3,b2,b1,b0)
           (f6,f5,f4,f3,f2,f1,f0) (g6,g5,g4,g3,g2,g1,g0)
	   =
        ((h3,h2,h1,h0,a6,a5,a4,a3,a2,a1,a0,b6,b5,b4,b3,b2,b1,b0,
         f6,f5,f4,f3,f2,f1,f0,g6,g5,g4,g3,g2,g1,g0):Int32);
fun split7 ((h3,h2,h1,h0,a6,a5,a4,a3,a2,a1,a0,b6,b5,b4,b3,b2,b1,b0,
         f6,f5,f4,f3,f2,f1,f0,g6,g5,g4,g3,g2,g1,g0):Int32)=
	((h3,h2,h1,h0),
	   (a6,a5,a4,a3,a2,a1,a0),(b6,b5,b4,b3,b2,b1,b0),
           (f6,f5,f4,f3,f2,f1,f0),(g6,g5,g4,g3,g2,g1,g0));
\end{verbatim}

These functions extract the source destination and index fields for use in instruction execution.
\begin{verbatim}
(* Decode.L
   ========
        Functions which  extract portions  of a number for address decoding
purposes 
2/1/89 sal *)
(* The bit indicating whether to index the source or not *)
val IndexX=dataBit31;
(* The source address *)
fun Source i=
        (dataBit30 i,dataBit29 i,dataBit28 i,dataBit27 i,
         dataBit26 i,dataBit25 i,dataBit24 i,dataBit23 i,
         dataBit22 i,dataBit21 i,dataBit20 i,dataBit19 i,
         dataBit18 i,dataBit17 i,dataBit16 i):Int15;
val IndexY=dataBit15;
fun Destination i=
        (dataBit14 i,dataBit13 i,dataBit12 i,dataBit11 i,
         dataBit10 i,dataBit9  i,dataBit8  i,dataBit7  i,
         dataBit6  i,dataBit5  i,dataBit4  i,dataBit3  i,
         dataBit2  i,dataBit1  i,dataBit0  i):Int15;
(* extract the l.s. 15 bits from a data word *)
val Truncate15=Destination;
(* expand a single bit to the full length of a data word *)
fun Expand b=(false,false,false,false,false,false,false,false,
                false,false,false,false,false,false,false,false,
                false,false,false,false,false,false,false,false,
                false,false,false,false,false,false,false,b):Int32;
\end{verbatim}

A few functions had to be added so that the following files could be parsed by either Lambda or ML.
A different copy of this file is used in the simulation which can not use Church' s iota function.
\begin{verbatim}
(* Spec.L	1.4
   ======
Special stuff put in to the Lambda specification alone  to make it more ML like
10/1/89 sal *)
fun not a=~a;
(* definition of successor-15 function avoiding binary maths *)
fun S15 i =
        if i == Maxint15 then Zero15 else
                iota j. Int15toNat j == S(Int15toNat i);

(* definition of successor-32 function avoiding binary maths *)
fun S32 i =
        if i == Maxint32 then Zero32 else
                iota j. Int32toNat j == S(Int32toNat i);
(* define time as measured in nanoSeconds*)
val nS=Natural;
\end{verbatim}

All variables have to be explicitly stated in Lambda, unlike ML.
\begin{verbatim}
(* variables.L
   ===========
Variables used in the specification  over  and  above  those  over and above
those normally provided 
2/1/89 sal *)
(* generate signals *)
vbl G,g0,g1,g2,g3,g4,g5,g6,g7,G0,G1     :bool;
(* propagate signals *)
vbl P,p0,p1,p2,p3,p4,p5,p6,p7,P0,P1     :bool;
(* carry bits *)
vbl carry_in,carry_out,c3,c7,c11,c15,c19,c23,c27        :bool;
(* zero flags *)
vbl z0,z1,z2,z3,z4;
(* general use (mainly in alu ) variables *)
vbl a0,a1,a2,a3,a4,a5,a6,a7,addr4;
vbl b0,b1,b2,b3,b4,b5,b6,b7;
vbl data0
vbl f,f0,f1,f2,f3,f4,f5,f6,f7,f28,f29,f30,f31;
vbl g,h,h0,h1,h2,h3,h4,h5,h6,h7,h28,h29,h30,h31;
vbl s0,s1,s2;
vbl shift;
\end{verbatim}

Each ALU operation is first described as a function operating between
two four bit tuples and a carry flag.
These are then combined into a single function to mimic the functionality of the ALU.
\begin{verbatim}
(* Math.L	1.5	5/24/89
   ======
Definition of Int32 Maths as performed by the ALU.
The operations are given as functions with no timing constraints.
3/1/89 sal
        *)
(* the exclusive or function *)
infix 6 xor;
fun a xor b= (a ||| b) && ~ (a && b);
(* A full adder using xor *)
fun full_add a b c=
        a xor b xor c;
(* how to add two bool 4 tuples using carry lookahead *)
fun add4 (a3,a2,a1,a0) (b3,b2,b1,b0) c=
  let   val g0=a0 && b0 
        val g1=a1 && b1 
        val g2=a2 && b2 
        val p0=a0 ||| b0
        val p1=a1 ||| b1
        val p2=a2 ||| b2
  in
        (full_add a3 b3 (g2 ||| (p2 && (g1 |||
                                (p1 && (g0 ||| (p0 && c)))))),
         full_add a2 b2 (g1 ||| (p1 && (g0 ||| (p0 && c)))),
         full_add a1 b1 (g0 ||| (p0 && c)),
	 full_add a0 b0 c)
           end;
fun preset4 _ _ _=(true,true,true,true);
fun clear4  _ _ _=(false,false,false,false);
fun and4 (a3,a2,a1,a0) (b3,b2,b1,b0) _ =
        (a3 && b3,a2 && b2, a1 && b1,a0 && b0);
fun or4 (a3,a2,a1,a0)  (b3,b2,b1,b0)_ =
        (a3 ||| b3,a2 ||| b2, a1 ||| b1,a0 ||| b0);
fun xor4 (a3,a2,a1,a0) (b3,b2,b1,b0) _ =
        (a3 xor b3,a2 xor b2, a1 xor b1,a0 xor b0);

fun not4 (a3,a2,a1,a0) = (~a3,~a2,~a1,~a0);

(* dont know exactly what this does yet *)

fun minus4 a b c=
        add4 a (not4 b) c;
(* describe how a different selction of the alu function produces
   different results *)
fun applyALU a b c false false false= clear4  a b c
 |  applyALU a b c false false true = minus4  a b c
 |  applyALU a b c false true  false= minus4  b a c
 |  applyALU a b c false true  true = add4   a b c
 |  applyALU a b c true  false false= xor4    a b c
 |  applyALU a b c true  false true = or4     a b c
 |  applyALU a b c true  true  false= and4    a b c
 |  applyALU a b c true  true  true = preset4 a b c;
fun propagate (a3,a2,a1,a0) (b3,b2,b1,b0)=
        ~(a0 && b0 && a1 && b1 && a2 && b2 && a3 && b3);
fun generate (a3,a2,a1,a0) (b3,b2,b1,b0)=
  let   val g0=a0 && b0 
        val g1=a1 && b1 
        val g2=a2 && b2 
        val g3=a3 && b3
        val p1=a1 ||| b1
        val p2=a2 ||| b2
        val p3=a3 ||| b3
  in
        ~(g3 |||  p3 && g2 ||| p3 && p2 && g1 
                          ||| p3 && p2 && p1 && g0)
  end;
\end{verbatim}

The components are individually specified as rewrite rules.
The delays between inputs and outputs are specifed as constants, to simplify
changing to different logic families.
\begin{verbatim}
     (* Components.L 
     -the formal specification of individual components
      -with timing in nS
     2/1/89 sal: SN74F182 Look ahead carry unit
     3/1/89 sal: SN74F381 ALU
        *)
     (* SN74F182 Look-ahead carry generators 
        ====================================
        These TTL  components take  as inputs  the P' and G' signals from
        either ALU's or other '182 units, to return a carry for each ALU/
        lookahead  unit,  and  Propagate  and  Generate  signals  for the
	combined units *)
     (* delay between p,g signals and G *)
     val t_pg_G=8;
     (* delay between p,g signals and P *)
     val t_pg_P=6;
     (* delay between carry in and carries out *)
     val t_c_c=5;
     val t_pg_c=5;
     (* difference in delays between t_c_c and t_pg_c *)
     val t_c_pg_c=t_c_c-t_pg_c;
     val SN74F182 #(g0,g1,g2,g3,p0,p1,p2,p3,c,c1,c2,c3,G,P)=
     (* g0-g3,p0-p3: generate and propagate inputs
        c: carry in
        c1,c2,c3: carry outputs
        G,P: Generate and Propagate outputs *)
        forall t:nS.
                (G (t+t_pg_G) == ~(g3 t || (p3 t && g2 t)
                        || (p3 t && p2 t && g1 t)
                        || (p3 t && p2 t && p1 t && g0)))
        /\
        forall t:nS.
                (P (t+t_pg_P) == ~(p0 t && p1 t && p2 t && p3 t))
        /\
        forall t:nS.
                (c1 (t+t_c_c) == g0 (t+t_c_pg_c) ||
                        (p0 (t+t_c_pg_c) && c t))
        /\
        forall t:nS.
                (c2 (t+t_c_c)== g1 (t+t_c_pg_c)||
                        (p1 (t+t_c_pg_c) && ((p0 (t+t_c_pg_c) && c t)
                                || g0 (t+t_c_pg_c))))
        /\
        forall t:nS.
                (c3 (t+t_c_pg_c)== g2 (t+t_c_pg_c)||
                        (p2 (t+t_c_pg_c) && (g1 (t+t_c_pg_c)||
                        (p1 (t+t_c_pg_c) && ((p0 (t+t_c_pg_c) && c t)
                                || g0 (t+t_c_pg_c)))));

     (* SN74F381 : ALU/function generator
        =================================
                These are the TTL i.c's which form the  core of  the ALU;
        each takes  two 4-bit  words ,  carry in  and 3  bits to select a
        function. Seven possible functions  can  be  performed: addition,
        subtraction, preset,clear,  and, or, exclusive or. One can select
        which word is be subtracted from the other.
        The units produce propagate and generate signals for  use by '182
        look-ahead carry units. *)
     (* typical propagation delays -from data sheet *)
     val t_c_f=8;
     val t_ab_g=7;
     val t_ab_p=7;
     val t_ab_f=11;
     val t_s_any=10;
     val t_ab_c_f=t_ab_f-t_c_f;
     val SN74F381 #(a,b,c,p,g,s2,s1,s0,f)=
     (* a,b:4-bit operands (input)
        c: carry in
        p,g:propagate and generate signals (output)
        s2,s1,s0: function select signals
        f: 4 bit result *)
        (a3,a2,a1,a0)=a /\
        forall t:nS.
                (p (t+t_ab_p)==propagate (a t) (b t))
        /\
        forall t:nS.
                (g (t+t_ab_g)==generate (a t) (b t))
        /\
        forall t:nS.
                (f (t+t_ab_f)==
                        applyALU (a t) (b t) (c (t+t_ab_c_f)
                                 (s2 t) (s1 t) (s0 t));

\end{verbatim}

All the pals were defined with timing constraints included.
\begin{verbatim}All the pals were defined with timing constraints included.
\begin{verbatim}
(* PALS.L
   ======
   Definition of PALS,EPLDS..etc.
   In ALU, address decoders, control...
3/1/89 sal :ALU function pals*)
(* delay for combinatorial PAL *)
val t_pal=25;
(* The programming for  four of the PALS in the Shift unit of the ALU *)
val ALU_shift_PAL #(shift,f7,(f6,f5,f4,f3,f2,f1,f0),
                z,(h6,h5,h4,h3,h2,h1,h0))=
        forall t:nS.
                (z (t+t_pal)==
                        (~(shift t) && ~(f0 t) && ~(f1 t) && ~(f2 t) &&
                        ~(f3 t) && ~(f4 t)  && ~(f5 t) && ~(f6 t))
                                 ||
                        (shift t && ~(f1 t) && ~(f2 t) && ~(f3 t) && ~(f4 t) 
                        && ~(f5  t) && ~(f6 t) && ~(f7 t)) 
        /\
        forall t:nS.
                (h0 (t+t_pal)== (f0 t) && ~(shift t) ||
                                (f1 t) && (shift t))
        /\
        forall t:nS.
                (h1 (t+t_pal)== (f1 t) && ~(shift t) ||
                                (f2 t) && (shift t))
        /\
        forall t:nS.
                (h2 (t+t_pal)== (f2 t) && ~(shift t) ||
                                (f3 t) && (shift t))
        /\
        forall t:nS.
                (h3 (t+t_pal)== (f3 t) && ~(shift t) ||
                                (f4 t) && (shift t))
        /\
        forall t:nS.
                (h4 (t+t_pal)== (f4 t) && ~(shift t) ||
                                (f5 t) && (shift t))
        /\
        forall t:nS.
                (h5 (t+t_pal)== (f5 t) && ~(shift t) ||
                                (f6 t) && (shift t))
        /\
        forall t:nS.
                (h6 (t+t_pal)== (f6 t) && ~(shift t) ||
                                (f7 t) && (shift t));

(* the shift PAL program for the most significant PAL *)

          fun ALU_SHIFT_PAL_fn_2  shift d0 c f31 f30 f29 f28
           =
                let val h28= ( ~shift && f28 ||| shift && f29) 
                and h29 = ~shift && f29 ||| shift && f30
                and h30 = ~shift && f30 ||| shift && f31
                and h31 = ~shift && f31 ||| shift && c
                and carry_out = ~shift && c ||| shift && d0
                and z = ~shift && ~f28 && ~f29 && ~f30 && ~f31
                        |||
                       shift && ~f29 && ~f30 && ~f31 && c
                in
                        (z,carry_out,(h31,h30,h29,h28))
                end;
val ALU_SHIFT_PAL_2 # shift d0 c f31 f30 f29 f28 z carry_out 
               (h31,h30,h29,h28)=
        forall t:nS.
               (z (t+t_pal),carry_out (t+t_pal),
               (h31 (t+t_pal),h30 (t+t_pal),
               h29 (t+t_pal),h28 (t+t_pal)))==
               ALU_SHIFT_PAL_fn_2  shift (d0 t) (c t) (f31 t) 
                                         (f30 t) (f29 t) (f28 t);
(* The Programming of the CC PAL in the ALU *)
val ALU_CC_PAL #(shift,z0,z1,z2,z3,z4,carry_in,data0,addr4
                 z,c)=
  forall t:nS.
        (z (t+t_pal)== (z0 t) && (z1 t) && (z2 t) && (z3 t)&& (z4 t))
  /\
   forall t:nS.
        (c (t+t_pal)== (carry_in t) && (addr4 t) ||
                        (data0 t) && (~addr4 t));
\end{verbatim}

The components can be combined to specify exactly how the ALU should behave.
This could, if desired, be verified against the mathematics of of a subset of integers.
\begin{verbatim}
(*              ALU.L
                =====

The definition of the ALU as a whole.

3/1/89 sal: TTL part only; not the PALS
*)

(* The carry look-ahead unit *)
(* built out of three TTL devices *)

val carrylookahead #( g0,g1,g2,g3,g4,g5,g6,g7,
                        p0,p1,p2,p3,p4,p5,p6,p7,
                        c,c3,c7,c11,c15,c19,c23,c27,carry_out)=

        SN74F182 #(g0,g1,g2,g3,p0,p1,p2,p3,c,c3,c7,c11,G0,P0)
        /\
        SN74F182 #(g4,g5,g6,g7,p4,p5,p6,p7,c15,c19,c23,c27,G1,P1)
        /\
        SN74F182 #(G0,G1,true,true,P0,P1,true,true,c,c15,carry_out,
                        false,true,true);


(* the TTL portion of the ALU *)
(* eight four bit slices connected together via the carry generator *)
(* a= Input a
   b= Input b
   c= carry in
   s2-s0= control signals
   c=carry out
   f=evaluated function
    *)


val ttlALU # (a b c s2 s1 s0 carry_out f)=
        (a7,a6,a5,a4,a3,a2,a1,a0)==split a /\
        (b7,b6,b5,b4,b3,b2,b1,b0)==split b /\
        carrylookahead #( g0,g1,g2,g3,g4,g5,g6,g7,
                        p0,p1,p2,p3,p4,p5,p6,p7,
                        c,c3,c7,c11,c15,c19,c23,c27,carry_out) /\
        SN74F381#(a0,b0,c,p0,g0,s2,s1,s0,f0) /\
        SN74F381#(a1,b1,c3,p1,g1,s2,s1,s0,f1) /\
        SN74F381#(a2,b2,c7,p2,g2,s2,s1,s0,f2) /\
        SN74F381#(a3,b3,c11,p3,g3,s2,s1,s0,f3) /\
        SN74F381#(a4,b4,c15,p4,g4,s2,s1,s0,f4) /\
        SN74F381#(a5,b5,c19,p5,g5,s2,s1,s0,f5) /\
        SN74F381#(a6,b6,c23,p6,g6,s2,s1,s0,f6) /\
        SN74F381#(a7,b7,c27,p7,g7,s2,s1,s0,f7) /\
        split f=(f7,f6,f5,f4,f3,f2,f1,f0);

\end{verbatim}

%\newpage
\section{ML simulation}
This extends the specification to describe instruction execution,
and wraps this in a monitor shell.
Some of the Lambda functions which only took a couple of lines have had to be translated into page-long ML operations.
These implement a few lambda functions in ML. 
Not having an {\bf iota} function, the successor functions are implemented by first defining functions converting from integers to Int15 and Int32.
The successor function is implemented by converting the boolean tuple to an integer representation, adding one, then converting it back.
The length of this file compared with the lambda specification indicates the differences in power of the two notations.

\begin{verbatim}
(* Extras.SIM  v1.9  5/24/89*) 
(* Extra functions needed for the simulation that Lambda doesn't 
   6/1/89 sal
   *)
           
          exception not_a_nibble; 

          (*convert a number from  0-16 to a binary equivalent*)

          fun to_nibble n= 
                if n>16 orelse n<0 then raise not_a_nibble
                else 
                        ((n div 8) =1, 
                         (n div 4 mod 2)=1, 
                         (n div 2 mod 2)=1, 
                         (n mod 2)=1); 

          (* convert any number to a list of nibbles *)

          fun NattoNibbles n = 
                if n<16 then [to_nibble n]  
                else NattoNibbles (n div 16) @ [to_nibble (n mod 16)] ;

          (* measure the length of a list *) 

          fun length nil = 0 
           |  length (_::t) = (length t) + 1; 

          (* extend a list to the lenth required *)

          exception list_too_long ;
           
          fun extend n L= if (length L) < n 
                        then ((false,false,false,false)::extend (n-1) L) 
                        else 
                        	if (length L) > n 
                        	then raise list_too_long
                        	else L; 

           
          val maxint15=Int15toNat Maxint15; 
           
          exception address_too_big ;  
           
		

          fun NattoInt15 n= 
                if n<= maxint15 then 
                        let val (a::b::c::d::_)= extend 4 ( NattoNibbles n) 
				in 
					(mergeInt15 a b c d):Int15
				end
                else raise address_too_big; 

	  fun S15 word=
		if word=Maxint15 then
			Zero15
		else
			NattoInt15 (1 + (Int15toNat word));

           
          val maxint32=Int32toNat Maxint32; 
           
          exception data_too_big ;
 
                fun NattoInt32 n= 
                if n<=maxint32 then  
			let val [a,b,c,d,e,f,g,h]=extend 8 (NattoNibbles n )
				in
					merge a b c d e f g h
				end
                else raise data_too_big; 

          infix 8 &&;

          fun a && b = a andalso b;

          infix 7 |||;
          fun a ||| b = a orelse b;

          fun ~ a = not a;

	fun plus15 a b=
		NattoInt15 ((Int15toNat a) + (Int15toNat b));
\end{verbatim}

All the register addresses are defined as constant 15-tuple addresses.
\begin{verbatim}
(*            CONSTANTS.SIM  v1.4	5/24/89
              =============  =====	======

                Constant  values  for  use  by  the simulation: these
                consist of mnemonics for memory addresses
           6/1/88 sal *)
          (* program counter *)
          val PC=Zero15;

          (* skip register *)
          val SKIP=S15  PC;

          (* X index *)
          val X=S15  SKIP;

          (* Y index *)
          val Y=S15  X;

          (* The number four in int15 format *)
          val PLUS4= S15 o S15 o S15 o S15 ;

          (*The accumulator on read *)
          val ACC=NattoInt15 8;

          (*which is the carry on write*)
          val CARRYIN=ACC;

          (*The accumulator Functions *)

          val CLR=NattoInt15 16;
          val SUBA=NattoInt15 17;
          val SUBB=NattoInt15 18;
          val ADD=NattoInt15 19;
          val XOR=NattoInt15 20;
          val OR=NattoInt15 21;
          val AND=NattoInt15 22;
          val SET=NattoInt15 23;

          (* The additional value to cause a shift *)
          val SHIFT=ACC;

          (* the condition codes *)
          val Z=CLR;
          val N=SUBA;
          val V=SUBB;
          val CARRY=ADD;
\end{verbatim}

The operating state of the Ultimate RISC is defined as a number of records.
One gives the state of the execution unit, another that of the ALU.
The entire machine is then represented as the two combined with a function to describe RAM.
\begin{verbatim}
(* 	states.SIM	v1.6
	============

         (* State description for the simulation *)

          type EXstate={pc:Int15, x:Int15, y:Int15,halt:bool};

          type ALUstate={acc:Int32,z:bool,n:bool,v:bool,carry:bool};

          type State={mem:Int15->Int32,
		exstate:EXstate,
		alustate:ALUstate};

          (* functions to extract individual fields}
          fun get_mem ({mem=m,...}:State)=m;
          fun get_ex ({exstate=e,...}:State)=e;
          fun get_alu ({alustate=a,...}:State)=a;
          fun get_pc ({pc=p,...}:EXstate) =p;
          fun get_x ({x=x,...}:EXstate) =x;
          fun get_y ({y=y,...}:EXstate) =y;
	  fun get_halt ({halt=h,...}:EXstate)=h;
          fun get_acc ({acc=a,...}:ALUstate) =a;
          fun get_carry ({carry=c,...}:ALUstate) =c;
          fun get_zero ({z=z,...}:ALUstate) =z;
          fun get_negative ({n=n,...}:ALUstate) =n;
          fun get_overflow ({v=v,...}:ALUstate) =v;
\end{verbatim} 

This describes the operation of the ALU.
Based upon the lambda specification, the components have been redescribed as
functions, and combined to describe the entire ALU's operation.
\begin{verbatim}
(*	alu.SIM		v1.8 *)
(*	=======		=====*)

         (* Simulation of the ALU *)

(* the ALU bit slice component*)

          fun SN74F182 g0 g1 g2 g3  p0 p1 p2 p3 c= (* c1 c2 c3 G P *)
          let
                val G =(g3  && (p3  ||| g2 )
                                && (p3  ||| p2  ||| g1 )
                                && (p3  ||| p2 ||| p1  ||| g0))
                and  P=p0  ||| p1  ||| p2  ||| p3
                and  c1= not(g0 && (p1 ||| (~ c)))
                and  c2=not(g1 && (p0 ||| (g0 && p1 ||| (~ c))))
                and  c3= not(g2 && (p2 ||| (g1 && p0 ||| 
					(g0 && p1 ||| (~ c)))))
          in
                (c1,c2,c3,G,P)
	  end;

(* the lookahead carry generator *)

          fun carrylookahead g0 g1 g2 g3 g4 g5 g6 g7
                           p0 p1 p2 p3 p4 p5 p6 p7 c=
          let 
                val (c3,c7,c11,G0,P0)=SN74F182 g0 g1 g2 g3 p0 p1 p2 p3 c
          and    G1=(g7  && (p7  ||| g6 )
                                && (p7  ||| p6  ||| g5 )
                                && (p7  ||| p6 ||| p5  ||| g4))
          and    P1=p4 ||| p5 ||| p6 ||| p7
          in
                let
                        val (c15,carry_out,_,_,_)=SN74F182 G0 G1 true true
                                                           P0 P1  true true
          							c
                in
                        let val (c19,c23,c27,_,_) = SN74F182 g4 g5 g6 g7
                                                             p4  p5  p6  p7
          							c15
                        in
                                (c3,c7,c11,c15,c19,c23,c27,carry_out)
                        end
                end
          end;


(* the TTL part of the ALU)
          fun ttlALU a b c s2 s1 s0=
          let
                val (a7,a6,a5,a4,a3,a2,a1,a0)=split a
          and    (b7,b6,b5,b4,b3,b2,b1,b0)=split b
          in 
                let
                        val (c3,c7,c11,c15,c19,c23,c27,carry_out)=
                                carrylookahead
                                (generate a0 b0)
                                (generate a1 b1)
                                (generate a2 b2)
                                (generate a3 b3)
                                (generate a4 b4)
                                (generate a5 b5)
                                (generate a6 b6)
                                (generate a7 b7)
                                (propagate a0 a0)
                                (propagate a1 a1)
                                (propagate a2 a2)
                                (propagate a3 a3)
                                (propagate a4 a4)
                                (propagate a5 a5)
                                (propagate a6 a6)
                                (propagate a7 a7)
                                c
                in (carry_out, merge 
                        (applyALU a7 b7 c27 s2 s1 s0)
                        (applyALU a6 b6 c23 s2 s1 s0)
                        (applyALU a5 b5 c19 s2 s1 s0)
                        (applyALU a4 b4 c15 s2 s1 s0)
                        (applyALU a3 b3 c11 s2 s1 s0)
                        (applyALU a2 b2 c7 s2 s1 s0)
                        (applyALU a1 b1 c3 s2 s1 s0)
                        (applyALU a0 b0 c s2 s1 s0))
                end
          end;

(* the shift pal program for the four least significant PALS *)

          fun ALU_SHIFT_PAL_fn shift f7 (f6,f5,f4,f3,f2,f1,f0)=
          let val z= (~shift && ~f0 && ~f1 && ~f2 && ~f3 && ~f4
                                  	&& ~f5 && ~f6 ) |||
                                (shift && ~f1 && ~f2 && ~f3 && ~f4 && ~f5
                                  && ~f6 && ~f7 )
          and   h0= ~shift && f0 ||| shift && f1
          and   h1= ~shift && f1 ||| shift && f2
          and   h2= ~shift && f2 ||| shift && f3
          and   h3= ~shift && f3 ||| shift && f4
          and   h4= ~shift && f4 ||| shift && f5
          and   h5= ~shift && f5 ||| shift && f6
          and   h6= ~shift && f6 ||| shift && f7

          in
                (z,(h6,h5,h4,h3,h2,h1,h0))
          end;

(* the shift PAL program for the most significant PAL *)

	  fun ALU_SHIFT_PAL_fn_2  shift d0 c f31 f30 f29 f28
	   =
		let val h28= ( ~shift && f28 ||| shift && f29) 
		and h29 = ~shift && f29 ||| shift && f30
		and h30 = ~shift && f30 ||| shift && f31
		and h31 = ~shift && f31 ||| shift && c
		and carry_out = ~shift && c ||| shift && d0
		and z = ~shift && ~f28 && ~f29 && ~f30 && ~f31
			|||
		       shift && ~f29 && ~f30 && ~f31 && c
		in
			(z,carry_out,(h31,h30,h29,h28))
		end;


(* the condition code generation *)
          fun ALU_CC_PAL_fn shift z0 z1 z2 z3 z4 carry_in data0 addr4=
          let
                val z=z0 && z1 && z2 && z3 && z4
               and c=carry_in && addr4 ||| data0 && ~addr4
          in
                (z,c)
          end;

(* the complete ALU *)
(* takes the current ALU state, the data on the bus and the 
value of the address bus to return an updated ALU state*)

          fun alu (alustate:ALUstate) d a=
          let
                val (c,f)=ttlALU (get_acc alustate)
                                 d
                                 (get_carry alustate)
                                 (addressBit2 a)
                                 (addressBit1 a)
                                 (addressBit0 a)
          	and shift=addressBit3 a
          in 
                let val ((f31,f30,f29,f28),f3,f2,f1,f0)=
				split7 f in
                        let val (z0,h0)=ALU_SHIFT_PAL_fn shift (dataBit7 f) f0
                            and (z1,h1)=ALU_SHIFT_PAL_fn shift (dataBit14 f) f1
                            and (z2,h2)=ALU_SHIFT_PAL_fn shift (dataBit21 f) f2
			    and (z3,h3)=ALU_SHIFT_PAL_fn shift (dataBit28 f) f3
                            and (z4,c2,(h31,h30,h29,h28))=
				ALU_SHIFT_PAL_fn_2 shift 
					(dataBit0 f) c f31 f30 f29 f28
                        in
                                let val h = merge7 (h31,h30,h29,h28) 
                                            h3 h2 h1 h0
                                in
                                        let val (z,carry)=
          					ALU_CC_PAL_fn shift
                                                z0 z1 z2 z3 z4 c
						(dataBit31 d)
						(addressBit4 a)
                                        in
                                          ({acc=h,
                                           z=z,
                                           n=dataBit31 h,
                                           v=carry,
                                           carry=carry}:ALUstate)
                                        end
                                end
                        end
                end
          end;
\end{verbatim}

The memory of the computer can be specified with some difficulty.
RAM is described by a curried function.
A number of boolean functions decode addresses and generate the halt signal.
The read function returns either a RAM location's contents, the accumulator or
a condition flag.
The write function is more complex, being able to update the entire machine state.
\begin{verbatim}
(*	memory.SIM	v1.7 	5/24/89 *)
(*	==========	====	======= *)
(* Simulated memory functions *)         
(* functional memory specification: very memory inefficient *)

	     fun RAM (d:Int32) mem (a:Int15) aa= 
	          (* taken from Lambda's examples *)
                if a=aa then d else mem aa; 
                
           (*reset state -not quite true*)
          fun RAM0 (_:Int15)=Zero32; 
           
          (* A function to test if an address is that of ram or not 
                i.e. returns True iff  bit 14 is True *)

          fun ram_address a=
                addressBit14  a;

          (* A function to test if an address references the alu-
                i.e is in the range of addresses 16-31
                -or equivalently bit 4 =True and it is not a ram address *)

          fun alu_address a=
                ~(ram_address  a) && (addressBit4  a);


          (*  a function to check if an address is valid for a read.
          a boolean value will be returned, true if an address is valid,
          false otherwise.  This is  used to  intercept reads on write only
          registers.
          Can be implemented as a PAL equation *)

          (* A function to read memory.  Will either  return the  data at a
          RAM 
          location or the contents of a memory mapped register*)

          (* invalid === the address is <=7 *)


          fun valid_address a=
                (ram_address  a) |||
                (alu_address  a) |||
                (addressBit3  a);

          exception can't_read; 

          fun read state a= 
                if ram_address a then get_mem state a 
                else if addressBit4 a then 
                  Expand ( if addressBit1 a then 
                   if addressBit0 a then get_carry (get_alu state) 
                                     else get_overflow (get_alu state) 
                    else if addressBit0 a then get_negative (get_alu state)

                         else get_zero ( get_alu state)) 
                else  
		 if addressBit3 a then get_acc (get_alu state)
		 else raise can't_read; 

          (* Write 
                This function takes a  state, an  address and  a data
                item. It  will then  use this item to update the RISC
                state. The program counter will always be  adjusted -
                normally being  incremented, but after a write to the
                PC then it will be  changed  to  the  supplied value.
                Other internal registers - X,Y,skip,Acc and Carry can
                also  be  written  to.  Writing  to   the  other  ALU
                addresses causes  the ALU  to be activated to perform
                an operation upon the Accumulator,  the carry and the
                data supplied.
                Writes to  a ram  address cause the data to be stored
                at the specifed location. *)

          fun write state a data=
           if a=PC orelse (a=PLUS4 PC) then     (* program counter*)
                ({       mem=get_mem state,
                        exstate={
                                pc=Truncate15 data,
                                x=get_x (get_ex state),
                                y=get_y (get_ex state),
                                halt=get_halt (get_ex state)},
                        alustate=get_alu state}:State)
           else
           if a=SKIP orelse (a=PLUS4 SKIP) then
                {       mem=get_mem state,
                        exstate= {pc=
				(if dataBit0 data then
					S15 (get_pc (get_ex state))
				else get_pc (get_ex state)),
                                x=get_x (get_ex state),
                                y=get_y (get_ex state),
                                halt=get_halt (get_ex state)},
                        alustate=get_alu state} 
           else
           if a=X orelse a=PLUS4 X then
                {       mem=get_mem state,
                        exstate= {
                                pc=get_pc (get_ex state),
                                x=Truncate15 data,
                                y=get_y (get_ex state),
                                halt=get_halt (get_ex state)},
                        alustate=get_alu state}
           else
           if a=Y orelse a=PLUS4 Y then
                {       mem=get_mem state,
                        exstate= {
                                pc=get_pc (get_ex state),
                                x=get_x (get_ex state),
                                y=Truncate15 data,
                                halt=get_halt (get_ex state)},
                        alustate=get_alu state}
           else
           if  ram_address a then
                {       mem=RAM data (get_mem state) a,
                        exstate=get_ex state,
                        alustate=get_alu state}
           else
           if alu_address a then
                {       mem = get_mem state,
                        exstate= get_ex state,
                        alustate= alu (get_alu state) data a
                        }
           else
                (* else write to carry flag *)
                (* warning: the state of the N,Z,V flags can't be predicted
                        after this operation: an ADD 0 operation should be
                        performed to reevaluate the other flags-
			an action which'll clear the carry *)
                {       mem = get_mem state,
			exstate=get_ex state,
                        alustate= let
				val { acc=_,z=z,n=n,v=v,carry=c}=
					alu (get_alu state) data a
				in
				{ acc=get_acc (get_alu state),
                                        z=z,
                                        n=n,
                                        v=v,
                                        carry=c}
				end
                        }; 
\end{verbatim}

The description of the Execution Unit completes the specification  of the Ultimate RISC.
A function {\bf execute} from {\bf State} to {\bf State} specifies and simulates the execution of a single instruction
\begin{verbatim}
(*	execute.SIM	v1.4
	===========		*)

         (* simulation of the execution unit *) 
           
           (* conditional index operation*)
          fun index a f i=if f then a else plus15 a i; 
           
          fun execute (state:State)= 
          let 
                val {mem=m, 
                     exstate={pc=pc,x=x,y=y,halt=halt}, 
                     alustate=a}=state 
          in 
                (* state if source fetch fails*)
                let val halted=({mem=m, 
                            exstate={pc=S15 pc, 
                                     x=x, 
                                     y=y, 
                                     halt=true}, 
                            alustate=a} :State)
                      (* state after the fetch of the first instruction---
                         the PC has been incremented *)
                    and ss={mem=m, 
                            exstate={pc=S15 pc, 
                                     x=x,
				     y=y,
				     halt=false}, 
                            alustate=a} 
                in 
                      (* do nothing if already halted *)
                        if halt then state 
                        (* fetch instruction*)
                        else if not (valid_address pc) then state 
                        else let 
                                val i=read state pc 
                             in 
                                (*fetch source operand*)
                                if not (valid_address (index (Source i)
          						(IndexX i) x)) 
                                then halted  
                                (*write to destination*)
                                else write ss 
                                   (index (Destination i) (IndexY i) y) 
                                   (read ss  
                                        (index (Source i) (IndexX i) x)) 
                            end 
                end 
          end; 

\end{verbatim}

To actually use the simulation some extra monitor functions were written.
These provide memory reading and writing, code assembly, register access and all the functions one should expect from a monitor.
The actual simulated computer is stored in a reference record,
which the monitor functions address.
\begin{verbatim}
(* monitor.SIM v1.6 	5/24/89 
   ================	======*)
 
          (* functions to control the simulation as the monitor should *) 
           
          
           
          (* give starting state *)
 
          val reset_state=({mem=RAM0, 
                          exstate={pc=Zero15,x=Zero15,
                                   y=Zero15,halt=false},                  
                          alustate={acc=Zero32,
                                    z=true,n=false,
                                    v=false,carry=false}}:State);
 
          (* describe machine state with a reference *) 
 
	  val URISC=ref reset_state;
           
          (* functions to display a state in an understandable form *) 
           
          val print15=makestring o Int15toNat; 
          val print32=makestring o Int32toNat; 
           
          val CR="\n"; 
           
          fun dissassemble n= 
                " MOVE"^ 
                (if IndexX n 
                        then "X" 
                        else "")^ 
                (if IndexY n then "Y " else " ")
                ^"("^
                (print15 (Source n))^ "," ^
                (print15 (Destination n))^
                 ")" ^ CR; 
           
           
          fun show (state:State)= 
          let val {mem=m, 
                exstate={pc=pc,x=x,y=y,halt=halt}, 
                alustate={acc=acc,z=z,n=n,v=v,carry=c}}=state 
          in 
		output(std_out,
                CR^"PC   ="^(print15 pc)^ 
                CR^"X    ="^(print15 x)^
                CR^"Y    ="^(print15 y)^ 
                CR^"halt ="^(makestring halt)^ 
                CR^"ACC  ="^(print32 acc)^ 
                CR^"(z,n,v,c)="^(makestring (z,n,v,c))^ 
                 (if not ( valid_address pc) then 
                        (CR^"PC at write-only address"^CR) 
                 else 
                        (CR ^ "Current Instruction="^ 
                         (dissassemble (read state pc)))))
		  
                 
          end; 
           
	(* examine current registers *)
          fun ex ()=show (!URISC); 
           
	(* read an address *)
          fun r a =Int32toNat (read (!URISC)  (NattoInt15 a)); 
          
	(* dump address count  - dumps memory locations *)
 
         fun dump _ 0 =() 
           |  dump a n = 
                (output(std_out,(CR^ 
                 (makestring a)^ 
                 "\t:\t"^
                 (makestring (r a))^ 
                 "\t=\t"^ 
                 (dissassemble (read (!URISC) ( NattoInt15 a)))));
		dump (a+1) (n-1)); 
           
          fun step ()= 
                URISC:=execute (!URISC); 
           
          (* step & show *) 
          fun ss ()=(step();ex()); 
           
          fun run()=  (* step until halted *) 
                if get_halt (get_ex  (!URISC)) then () 
                else (step();run()); 
           
          fun set_pc pc= 
                URISC:={mem=get_mem (!URISC), 
                        exstate={pc=( NattoInt15 pc), 
                                 x=get_x (get_ex (!URISC)), 
                                 y=get_y (get_ex (!URISC)), 
                                 halt=get_halt ( get_ex (!URISC))}, 
                        alustate=get_alu (!URISC)};

          fun set_x x= 
                URISC:={mem=get_mem (!URISC), 
                        exstate={pc=get_pc ( get_ex (!URISC)), 
                                 x=( NattoInt15 x),
                                 y=get_y ( get_ex (!URISC)), 
                                 halt=get_halt ( get_ex (!URISC))}, 
                        alustate=get_alu (!URISC)}; 

          fun set_y y= 
                URISC:={mem=get_mem (!URISC), 
                        exstate={pc=get_pc (get_ex (!URISC)), 
                                 x=get_x ( get_ex (!URISC)), 
                                 y=( NattoInt15 y),
                                 halt=get_halt ( get_ex (!URISC))}, 
                        alustate=get_alu (!URISC)};

          fun set_halt  halt= 
                URISC:={mem=get_mem (!URISC), 
                        exstate={pc=get_pc ( get_ex (!URISC)), 
                                 x=get_x (get_ex (!URISC)), 
                                 y=get_y (get_ex (!URISC)), 
                                 halt=halt},
                        alustate=get_alu (!URISC)};
          fun set_acc a= 
                URISC:={mem=get_mem (!URISC), 
                        exstate=get_ex (!URISC), 
                        alustate={acc=( NattoInt32 a),
                                  z=get_zero ( get_alu (!URISC)),
                                  n=get_negative ( get_alu (!URISC)),
                                  v=get_overflow ( get_alu (!URISC)),
                                  carry=get_carry ( get_alu (!URISC))}};
         fun set_carry a= 
                URISC:={mem=get_mem (!URISC), 
                        exstate=get_ex (!URISC), 
                        alustate={acc=get_acc (get_alu (!URISC)),
                                  z=get_zero ( get_alu (!URISC)),
                                  n=get_negative ( get_alu (!URISC)),
                                  v=get_overflow ( get_alu (!URISC)),
                                  carry=a}};


          (* go until halted *) 
          fun go a= 
                (set_pc  a; 
                 set_halt false; 
                 run(); 
                 ex());  
           
          (* write number to an address *) 
          fun w address data= 
                URISC:=write (!URISC) (NattoInt15 address) (NattoInt32 data); 
          (* infix version of above *)

          infix 6 ##;
          fun a ## d=w a d;
	
	  val offset=65536;

          (* assemble a move instruction into memory *)
          infix 6 MOVE ; 
          fun a MOVE (s ,d)=w a (offset*s +d); 

          val INDEX=32768;

          infix 6 MOVEX ;
          fun a MOVEX (s, d)=a ## (offset*(INDEX+s)+d);

          infix 6 MOVEY ;
          fun a MOVEY (s ,d)=a ## (offset*s+d+INDEX);

          infix 6 MOVEXY ;
          fun a MOVEXY (s, d)=a ## (offset*(INDEX+s)+d+INDEX);

	(* Now the Register Addresses in Integer Format *)
	val PC=0;
	val SKIP=1;
	val X=2;
	val Y=3;
	val A=8;
	val CIN=8;

	(*The accumulator Functions *)

          val CLR= 16;
          val SUBA= 17;
          val SUBB= 18;
          val ADD= 19;
          val XOR= 20;
          val OR= 21;
          val AND= 22;
          val SET= 23;

	(* extra offset to cause post-rotate *)
	val SHIFT=8;

	(* the condition codes *)
          val Z=CLR;
          val N=SUBA;
          val V=SUBB;
          val CARRY=ADD;




          (* can assemble  move instuctions e.g *)
          (* a short loop *)

          8192 MOVE (9000, CLR);
          8193 MOVE (9004 ,ADD);
          8194 MOVE (A, X);
          8195 MOVE (A, Y);
          8196 MOVEXY (9000, 10000);
          8197 MOVE (9001,CIN);
          8198 MOVE (9001, SUBB);
          8199 MOVE (N ,SKIP);
          8200 MOVE (9005, PC);
          8201 MOVE (0,0); (* halt *)           
          9000  ## 0;
          9001  ## 1;
          9002  ## 2;
          9003  ## 3;
          9004  ## 4;
          9005  ## 8194;
\end{verbatim}

