
Research is continuously under way to try and build the fastest computers with the  technology available.
Low cost and high reliability are often of as high a priority as performance.
A recent concept is that of RISC computers, which  increase performance by reducing the number of instructions the computer can understand.

The aim of this  project was to take this architecture to its natural conclusion, to build a computer capable of executing only a single instruction.
It has demonstrated that such a machine can be easily and cheaply built, yet may
be compared, in terms of raw computing power, with commercial microprocessors.

The architecture of the computer was described mathematically,   a technique known as {\em Formal Specification}, and is one of the few computers to have been  so designed.
This technique  increases the likely reliability of a computer, but severely limits its complexity.
The project therefore provides an example into the formal specification of computer systems, and a demonstration of the the  associated difficulties.


The computer implemented is a 32-bit processor with an integer ALU and 32 Kilobytes of memory, built out of simple MSI Integrated Circuits.
It fits onto a single APM card, and can be connected to such a workstation via a co-processor.
A monitor program has been written to control the Ultimate RISC, providing facilities for hardware and software development.

It should be capable of executing approximately 3.3 Million instructions every second, although this has not been achieved for a number of reasons.


