\section{The Monitor}

The simplicity of the host interface is such that a sophisticated monitor 
program is needed to make any use of the Ultimate RISC.
The development of this monitor program went hand in hand with the building 
of the computer itself.
When a new feature was built into the hardware, it was first tested with the
existing monitor features, and once the protocol was established, built 
into higher level routines.

The facilities offered include
\begin{itemize}
\item loading and executing programs on the 6809 board.
\item manual manipulation of the host interface lines.
\item reading, modifying and copying back the registers of the Ultimate RISC.
\item single stepping the computer's clock.
\item reading and writing memory locations ---including registers
\item memory testing
\item downloading code into the Ultimate RISC's memory
\item executing single instructions, or until halted.
\item printing characters on behalf of the Ultimate RISC -emulated I/O
\end{itemize}

The monitor is divided into two programs, each executing 
concurrently on separate processors.

\subsection{A Virtual PIA}
The Ultimate RISC's host interface is  connected to the Peripheral Interface
Adaptor of the APM 6809 Processor Board. While most of the memory is shared 
with the 68000 processor this is not the case for I/O devices.
A short 6809 Assembly Language program is used to provide a virtual PIA for 
the 68000 microprocessor to use. 
This copies the input port of 
the PIA to a shared memory location, and uses data in other
 memory locations to update the PIA registers.

The first three memory locations of the 6809 processor's address space are dedicated to this inter-processor communications:-
\begin{description}
\item[0] : the data to be written
\item[1] : the PIA register(0--3) to write to, +128\\
set to zero after the write has been completed
\item[2] : the inputs to PIA port A
\end{description}
To use the PIA this virtual PIA program must be executing. 
The 68000 program must write the data first, followed by the register number+128.
This high bit is used to indicate a valid write.
When the contents of address 1 are cleared, the request has been processed and the copy of port A updated. The 68000 can use this to poll for the completion of an operation, and test the operation of the 6809 board.

This method of synchronisation in not particularly fast.
When controlling the clock from the host, the maximum clock speed is only 1\ KHz.

\subsection{The Main Monitor}
This is an extension of the CS2 6809 monitor, from which the Command Line Interface and the 6809 control operations are all taken.
It has been extended to form a 2000 line IMP program with many more commands.
There is also a status display at the top of the screen.
This shows the state of all the inputs and outputs on the host interface and 
the state of the control unit   as a mnemonic.

To provide flexibility in development most of the  data about registers and states are kept in separate files. 
This allows different configurations of both the control unit and of the registers to used with minimal effort.

The only problem with the monitor is that it is somewhat isolated from the Ultimate RISC. 
One could increase the speed of certain sections by providing increased support from the 6809 portion of the monitor, which is the hardest part to write and debug.
The registers which can be examined and modified are only copies of the actual registers, and one has to remember to copy these registers backwards and forwards.
The status line is of more use to hardware development than for software.
However, the monitor does provide access to the Ultimate RISC, and although inefficient and unwieldy, it is better than no monitor at all. 


\subsubsection{Code File Format}
The format of files which can be downloaded is designed to support both
object code produced by the compiler and hand coding:-
\begin{enumerate}
\item The file is stored as a text file, with no restrictions on naming
\item all numbers are given in hexadecimal, one to a line
\item any line beginning with `{\bf !}' is a comment, to be printed during
downloading
\item code sequences are represented as:-\\
address\\
code length\\
(code)\\
\item There is no limit upon the number of code sequences in a single file
\end{enumerate}
The downloading process is extremely slow at only four words per second, so the use of comments provides essential feedback as to the state of the transfer.

