
\chapter{Construction}

This appendix describes the actual construction of the Ultimate RISC.
The layout of the board is detailed, followed by a list of construction problems.

\section{Layout}

Building a board to operate at high speed, it was of the utmost importance to minimise the lengths of connecting wires.
This reduced the propagation delay and the likiehood of cross-talk or transmission line effects.

TTL devices can draw a large amount of power, and so each IC had to be decoupled with a small capacitor across the power and ground pins. 
The entire board also needed decoupling in the form of a larger capacitor across the power supply inputs.

The computer was designed to connect to a M6809 board within an APM cabinet.
This fixed the construction area to that of a double size Eurocard.
The  wire wrap board provided subsurface power and ground planes to reduce noise and ease power distribution.
With seventy ICs to place on this board careful packing was necessary.

Wire wrap sockets  were used to hold all ICs. This enabled easy insertion and removal of devices and permitted wire wrapping.

The connection between the Ultimate RISC and the Real Time Board was via  a Eurocard edge connector. 
This had to placed at the bottom rear of the board ---as seen from the front of the cabinet--- to enable it to be connected to the PIA ports.
It was this fact which determined the layout of the rest of the board (figure~\ref{fig:layout}).

\begin{figure}
\vspace{10cm}
\caption{Board Layout}
\label{fig:layout}
\end{figure}

The host inputs had to be latched and then passed to the Control Unit, which had therefore to be adjacent to the connector and latches.
The common clock was a highly critical signal, and all devices which received it needed to be as close together as possible.
This placed the PC,  the address decoder and the clock oscillator all near to the Control Unit.

The Execution Unit consisted mainly of sixteen registers. 
These were easiest lain out together in a large block, achieving high density and aiding wiring of the SSR chain.
The unit most tightly coupled to the  Control Unit, it was placed immediately above this EPLD.
The operand index addition was performed by a row of ICs above these registers.

The two other units needing placement were the ALU and the RAM.
The RAM was only four, albeit large, chips, and were easily fitted into a spare 
area at the base of the board, next to the address decoder.

The ALU was allocated the top third of the card, with its interface to the data bus at the base of this region.
The longest distance which signals travel to reach this unit is approximately 15cm, which should cause an extra delay of under a nanosecond.

Two light emitting diodes were placed above the execution unit, where they are easily visible.
These provide feedback as to the operation of the board, lighting up when power is supplied and flickering during operation.
The top LED emits a green light whenever the board is not halted.
The lower LED is connected to the clock signal. 
It flashes whenever data is being transferred between this boad and the host, and is bright red during instruction execution.

A second edge connector was inserted into the board. 
This was purely to hold the board in place while being tested on an extender card.

\section{Power Supply}

The Ultimate RISC   draws four Amps of current, a power consumption of 20~Watts.
This is mainly due  to the requirements of the AMD registers, which contain ECL logic internally. 
This power has to be dissipated by these devices, which can become almost too hot to touch.
The power consumption is too large to be supplied by the single thin wire from the 6809 board.
Instead the Ultimate RISC has to be connected to an external power supply.
This also provides a convenient method of resetting the  computer.
It also created a few unforseen problems.

\section{Construction Problems}
The following problems all caused delays during construction.

\subsubsection{Faulty EPLD programming}
Mistakes made in the design of the EPLD programs for the Control Unit and address decoder caused regular problems. 
Some mistakes caused the computer to behave unreliably, or only partly correctly.
For example, an early version of the control unit moved the source to source, an operation which is not of much use.
The EPLD Software also lacked reliability. 
This is most apparent when using the state machine design program.
This program ignores the definition of  output transitions for a particular state, and yet will not produce a state machine at all if this state is not included.
I was forced to retain this state but bypass it during  instruction execution.
Sometimes, however, the EPLD suffers from metastability problems, and in these situations the state can be reached; a power down reset is then obligatory.

\subsubsection{Faulty Wiring}
A few wrongly connected wires created difficulties.
An accidental short of the output enable control of the Data register prevented memory read accesses from working for a whole week. 
Mis-wired power and ground connections on an IC also melted a buffer IC; a replacement
was easily obtained.

\subsubsection{Power Supply Difficulties}
At one stage I was confused for a number of days by the nondeterministic behaviour of the computer. 
I suspected this was due to EPLD problems, and wasted much time trying to verify their operation.
In fact, as was pointed out to me, an overloaded  power supply was only producing
a potential of 3.7~Volts between power and ground.

Upgrading to a higher rated power supply solved this problem, but created a new one.
One one occasion this power supply had been switched on while repeated attempts were made to boot up an unreliable APM. 
A current loop must have been created as the power supply attempted to drive the entire APM via the ICs on my computer.
I noticed the problem by the smell of solder, and then proceeded to burn my hand upon an AMD SSR. 
Suprisingly, these registers still worked later.

\subsubsection{Avoiding Construction Problems}
It is worthwhile studying these problems to see if the application of formal methods on a larger scale could have prevented them.

Mistakes in the EPLD programs which I had made myself could have been eliminated if the operation of the computer at this microstate level had been specified and verifed against the higher level specificatio.
It could not prevent the problems caused by the EPLD programmer itself.
If there was enough confidence in the state machine specifications, then perhaps these could have been implemented in a PAL instead.

Faulty wiring can be prevented by having  a machine which can reliably wire up a correct design. There do not seem to be one of these available within the department;
even the BEPI robots's products have to be checked.

Formal Methods would  not have provided any defence against the power supply problems; the devices were operating outside their stated limits. 
If I had had more confidence in my state machine design I may have come to suspect the power supply much earlier.

Note that there were no problems  with the construction of the those parts of the ALU which were fully specified ---the mathematics seem to be as predicted.


