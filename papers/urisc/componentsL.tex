The components are individually specified as rewrite rules.
The delays between inputs and outputs are specifed as constants, to simplify
changing to different logic families.
\begin{verbatim}
     (* Components.L 
     -the formal specification of individual components
      -with timing in nS
     2/1/89 sal: SN74F182 Look ahead carry unit
     3/1/89 sal: SN74F381 ALU
        *)
     (* SN74F182 Look-ahead carry generators 
        ====================================
        These TTL  components take  as inputs  the P' and G' signals from
        either ALU's or other '182 units, to return a carry for each ALU/
        lookahead  unit,  and  Propagate  and  Generate  signals  for the
	combined units *)
     (* delay between p,g signals and G *)
     val t_pg_G=8;
     (* delay between p,g signals and P *)
     val t_pg_P=6;
     (* delay between carry in and carries out *)
     val t_c_c=5;
     val t_pg_c=5;
     (* difference in delays between t_c_c and t_pg_c *)
     val t_c_pg_c=t_c_c-t_pg_c;
     val SN74F182 #(g0,g1,g2,g3,p0,p1,p2,p3,c,c1,c2,c3,G,P)=
     (* g0-g3,p0-p3: generate and propagate inputs
        c: carry in
        c1,c2,c3: carry outputs
        G,P: Generate and Propagate outputs *)
        forall t:nS.
                (G (t+t_pg_G) == ~(g3 t || (p3 t && g2 t)
                        || (p3 t && p2 t && g1 t)
                        || (p3 t && p2 t && p1 t && g0)))
        /\
        forall t:nS.
                (P (t+t_pg_P) == ~(p0 t && p1 t && p2 t && p3 t))
        /\
        forall t:nS.
                (c1 (t+t_c_c) == g0 (t+t_c_pg_c) ||
                        (p0 (t+t_c_pg_c) && c t))
        /\
        forall t:nS.
                (c2 (t+t_c_c)== g1 (t+t_c_pg_c)||
                        (p1 (t+t_c_pg_c) && ((p0 (t+t_c_pg_c) && c t)
                                || g0 (t+t_c_pg_c))))
        /\
        forall t:nS.
                (c3 (t+t_c_pg_c)== g2 (t+t_c_pg_c)||
                        (p2 (t+t_c_pg_c) && (g1 (t+t_c_pg_c)||
                        (p1 (t+t_c_pg_c) && ((p0 (t+t_c_pg_c) && c t)
                                || g0 (t+t_c_pg_c)))));

     (* SN74F381 : ALU/function generator
        =================================
                These are the TTL i.c's which form the  core of  the ALU;
        each takes  two 4-bit  words ,  carry in  and 3  bits to select a
        function. Seven possible functions  can  be  performed: addition,
        subtraction, preset,clear,  and, or, exclusive or. One can select
        which word is be subtracted from the other.
        The units produce propagate and generate signals for  use by '182
        look-ahead carry units. *)
     (* typical propagation delays -from data sheet *)
     val t_c_f=8;
     val t_ab_g=7;
     val t_ab_p=7;
     val t_ab_f=11;
     val t_s_any=10;
     val t_ab_c_f=t_ab_f-t_c_f;
     val SN74F381 #(a,b,c,p,g,s2,s1,s0,f)=
     (* a,b:4-bit operands (input)
        c: carry in
        p,g:propagate and generate signals (output)
        s2,s1,s0: function select signals
        f: 4 bit result *)
        (a3,a2,a1,a0)=a /\
        forall t:nS.
                (p (t+t_ab_p)==propagate (a t) (b t))
        /\
        forall t:nS.
                (g (t+t_ab_g)==generate (a t) (b t))
        /\
        forall t:nS.
                (f (t+t_ab_f)==
                        applyALU (a t) (b t) (c (t+t_ab_c_f)
                                 (s2 t) (s1 t) (s0 t));

\end{verbatim}
