\chapter{The Execution Unit}

\section{Design}

Within this unit instructions are fetched and executed,
under the supervision of the control unit. It is 
depicted in figure~\ref{fig:ex}.
It  contains a number of registers in order to carry out this
task.
A 15 bit {\bf Program Counter} ({\bf PC}) is used 
to locate the next instruction.
 This 
is incremented after each instruction has been fetched. 
To permit program branching, this register is a writeable  
memory location.

There are two index registers ---the source index {\bf X} and the 
destination index {\bf Y}.  Both are 15 bits wide and  
memory mapped. 
A fast adder is used to add the contents of the registers to the 
operand addresses. 


Conditional branching is facilitated by  a 
{\bf Skip} register. This appears as a single bit memory-mapped register.
When written to with the least significant bit of the data  set, this causes the 
next instruction  to be skipped. 
If the bit is clear the following instruction is executed as normal.
 Without such a register
conditional branching could still be performed by placing a conditional offset into the source index register, and moving the contents of the resulting address to the PC.
 This would be more cumbersome for a simple branch, but effective in multiway branches.


\begin{figure}
\vspace{20cm}
\caption{Execution Unit}
\label{fig:ex}
\end{figure}

{\samepage
Internal registers are used during the execution of an 
instruction. These are not accessible by programs, but can be 
read and possibly altered by the host. They are :-
\begin{itemize}
\item
{\bf data} : 32 bits for storage of data while being moved
\item
{\bf address} : 15 bits for buffering of the current location being 
accessed.
\item
{\bf instruction }: 32 bits for storage of the current instruction.
It is subdivided into the {\bf source} and {\bf destination}, each of which
contains an {\bf index} flag and a 15 bit {\bf operand}
\end{itemize}
Their use is shown at the register transfer level in figure~\ref{figure:rtl}.
}


\begin{figure}



\begin{enumerate}
\item {\bf address} $\leftarrow$ {\bf PC}
\item {\bf instruction} $\leftarrow$ ({\bf address});\\
{\bf PC}$\leftarrow$ {\bf PC}+1
\item if {\bf instruction.source.index}=1 then \\
{\bf address}$\leftarrow$ {\bf instruction.source.operand} + {\bf X}\\
else\\
{\bf address}$\leftarrow$ {\bf instruction.source.operand}
\item {\bf data}$\leftarrow$({\bf address});\\
if {\bf instruction.destination.index}=1 then \\
{\bf address}$\leftarrow$ {\bf instruction.destination.operand} + {\bf Y}\\
else\\
{\bf address}$\leftarrow$ {\bf instruction.destination.operand}
\item ({\bf address})$\leftarrow$ {\bf data}
\end{enumerate}
\caption{Instruction Execution Sequence}
\label{figure:rtl}
\end{figure}

\section{Implementation}

\subsection{Registers}

All registers except the PC are  constructed out of Serial Shadow 
Registers. 
The {\bf Program Counter} is  recorded in a bank of parallel loading counters, which 
 can 
be incremented by a control signal and  
reloaded by a memory write. 
To allow the host the ability to read the {\bf PC} it is passed through a 
Shadow Serial Register during the instruction fetch sequence ---the
 {\bf Program Counter Register}
({\bf PCR}). 

No Execution Unit registers can be examined with a memory read operation. This is inconvenient, as a program cannot determine the contents of the {\bf PC}, {\bf X} or {\bf Y} registers.
This prevents relocatable code being used, and complicates other operations. Supporting the reading of  these registers would have used an extra eight tristate buffers, for which there was neither room nor money.

\subsection{Connections}

The outputs of the {\bf PCR}  and  the instruction operands are 
all connected to the inputs of the {\bf Address} register. 
An adder made from four 74381 ALU units 
and a carry lookahead generator  performs the adding of offsets to 
indexed instructions. 
The {\bf Source} and {\bf Destination}
operands and the {\bf X }
and {\bf Y} index registers are multiplexed into this adder using the tristate 
outputs of the registers.
The outputs of the adder are fed to a tristate buffer.
Another buffer exists to bypass the adder completely, for non-indexed operands.
The selection of whether to index the offset or not is controlled by the index flag of each operand, without   the control unit's intervention.


\subsection{Skipping} 

Instruction skipping is performed independently of the control unit.
The input to the count signal of the {\bf PC} is multiplexed between a count signal from the control section and bit zero of the data bus.
Signal selection is controlled by the address decoder. 
On recognising a write to the {\bf Skip} register this decoder connects
the databus to the count input for one clock period only.
The program counter  is then incremented only if the least significant bit if the data word is set.
