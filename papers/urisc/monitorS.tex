To actually use the simulation some extra monitor functions were written.
These provide memory reading and writing, code assembly, register access and all the functions one should expect from a monitor.
The actual simulated computer is stored in a reference record,
which the monitor functions address.
\begin{verbatim}
(* monitor.SIM v1.6 	5/24/89 
   ================	======*)
 
          (* functions to control the simulation as the monitor should *) 
           
          
           
          (* give starting state *)
 
          val reset_state=({mem=RAM0, 
                          exstate={pc=Zero15,x=Zero15,
                                   y=Zero15,halt=false},                  
                          alustate={acc=Zero32,
                                    z=true,n=false,
                                    v=false,carry=false}}:State);
 
          (* describe machine state with a reference *) 
 
	  val URISC=ref reset_state;
           
          (* functions to display a state in an understandable form *) 
           
          val print15=makestring o Int15toNat; 
          val print32=makestring o Int32toNat; 
           
          val CR="\n"; 
           
          fun dissassemble n= 
                " MOVE"^ 
                (if IndexX n 
                        then "X" 
                        else "")^ 
                (if IndexY n then "Y " else " ")
                ^"("^
                (print15 (Source n))^ "," ^
                (print15 (Destination n))^
                 ")" ^ CR; 
           
           
          fun show (state:State)= 
          let val {mem=m, 
                exstate={pc=pc,x=x,y=y,halt=halt}, 
                alustate={acc=acc,z=z,n=n,v=v,carry=c}}=state 
          in 
		output(std_out,
                CR^"PC   ="^(print15 pc)^ 
                CR^"X    ="^(print15 x)^
                CR^"Y    ="^(print15 y)^ 
                CR^"halt ="^(makestring halt)^ 
                CR^"ACC  ="^(print32 acc)^ 
                CR^"(z,n,v,c)="^(makestring (z,n,v,c))^ 
                 (if not ( valid_address pc) then 
                        (CR^"PC at write-only address"^CR) 
                 else 
                        (CR ^ "Current Instruction="^ 
                         (dissassemble (read state pc)))))
		  
                 
          end; 
           
	(* examine current registers *)
          fun ex ()=show (!URISC); 
           
	(* read an address *)
          fun r a =Int32toNat (read (!URISC)  (NattoInt15 a)); 
          
	(* dump address count  - dumps memory locations *)
 
         fun dump _ 0 =() 
           |  dump a n = 
                (output(std_out,(CR^ 
                 (makestring a)^ 
                 "\t:\t"^
                 (makestring (r a))^ 
                 "\t=\t"^ 
                 (dissassemble (read (!URISC) ( NattoInt15 a)))));
		dump (a+1) (n-1)); 
           
          fun step ()= 
                URISC:=execute (!URISC); 
           
          (* step & show *) 
          fun ss ()=(step();ex()); 
           
          fun run()=  (* step until halted *) 
                if get_halt (get_ex  (!URISC)) then () 
                else (step();run()); 
           
          fun set_pc pc= 
                URISC:={mem=get_mem (!URISC), 
                        exstate={pc=( NattoInt15 pc), 
                                 x=get_x (get_ex (!URISC)), 
                                 y=get_y (get_ex (!URISC)), 
                                 halt=get_halt ( get_ex (!URISC))}, 
                        alustate=get_alu (!URISC)};

          fun set_x x= 
                URISC:={mem=get_mem (!URISC), 
                        exstate={pc=get_pc ( get_ex (!URISC)), 
                                 x=( NattoInt15 x),
                                 y=get_y ( get_ex (!URISC)), 
                                 halt=get_halt ( get_ex (!URISC))}, 
                        alustate=get_alu (!URISC)}; 

          fun set_y y= 
                URISC:={mem=get_mem (!URISC), 
                        exstate={pc=get_pc (get_ex (!URISC)), 
                                 x=get_x ( get_ex (!URISC)), 
                                 y=( NattoInt15 y),
                                 halt=get_halt ( get_ex (!URISC))}, 
                        alustate=get_alu (!URISC)};

          fun set_halt  halt= 
                URISC:={mem=get_mem (!URISC), 
                        exstate={pc=get_pc ( get_ex (!URISC)), 
                                 x=get_x (get_ex (!URISC)), 
                                 y=get_y (get_ex (!URISC)), 
                                 halt=halt},
                        alustate=get_alu (!URISC)};
          fun set_acc a= 
                URISC:={mem=get_mem (!URISC), 
                        exstate=get_ex (!URISC), 
                        alustate={acc=( NattoInt32 a),
                                  z=get_zero ( get_alu (!URISC)),
                                  n=get_negative ( get_alu (!URISC)),
                                  v=get_overflow ( get_alu (!URISC)),
                                  carry=get_carry ( get_alu (!URISC))}};
         fun set_carry a= 
                URISC:={mem=get_mem (!URISC), 
                        exstate=get_ex (!URISC), 
                        alustate={acc=get_acc (get_alu (!URISC)),
                                  z=get_zero ( get_alu (!URISC)),
                                  n=get_negative ( get_alu (!URISC)),
                                  v=get_overflow ( get_alu (!URISC)),
                                  carry=a}};


          (* go until halted *) 
          fun go a= 
                (set_pc  a; 
                 set_halt false; 
                 run(); 
                 ex());  
           
          (* write number to an address *) 
          fun w address data= 
                URISC:=write (!URISC) (NattoInt15 address) (NattoInt32 data); 
          (* infix version of above *)

          infix 6 ##;
          fun a ## d=w a d;
	
	  val offset=65536;

          (* assemble a move instruction into memory *)
          infix 6 MOVE ; 
          fun a MOVE (s ,d)=w a (offset*s +d); 

          val INDEX=32768;

          infix 6 MOVEX ;
          fun a MOVEX (s, d)=a ## (offset*(INDEX+s)+d);

          infix 6 MOVEY ;
          fun a MOVEY (s ,d)=a ## (offset*s+d+INDEX);

          infix 6 MOVEXY ;
          fun a MOVEXY (s, d)=a ## (offset*(INDEX+s)+d+INDEX);

	(* Now the Register Addresses in Integer Format *)
	val PC=0;
	val SKIP=1;
	val X=2;
	val Y=3;
	val A=8;
	val CIN=8;

	(*The accumulator Functions *)

          val CLR= 16;
          val SUBA= 17;
          val SUBB= 18;
          val ADD= 19;
          val XOR= 20;
          val OR= 21;
          val AND= 22;
          val SET= 23;

	(* extra offset to cause post-rotate *)
	val SHIFT=8;

	(* the condition codes *)
          val Z=CLR;
          val N=SUBA;
          val V=SUBB;
          val CARRY=ADD;




          (* can assemble  move instuctions e.g *)
          (* a short loop *)

          8192 MOVE (9000, CLR);
          8193 MOVE (9004 ,ADD);
          8194 MOVE (A, X);
          8195 MOVE (A, Y);
          8196 MOVEXY (9000, 10000);
          8197 MOVE (9001,CIN);
          8198 MOVE (9001, SUBB);
          8199 MOVE (N ,SKIP);
          8200 MOVE (9005, PC);
          8201 MOVE (0,0); (* halt *)           
          9000  ## 0;
          9001  ## 1;
          9002  ## 2;
          9003  ## 3;
          9004  ## 4;
          9005  ## 8194;
\end{verbatim}
