\section{Ultimate RISCs}

 In 1956 W. van der Poel  showed how a single instruction was computationally sufficient for all programming needs. 
  Describing the instruction set of an existing computer, he
 demonstrated by a process of elimination that  only   one instruction  was actually needed \cite{poel:urisc}.
 The instruction was to subtract a value in memory from an accumulator, rewriting the result back to the memory location.
 If the number subtracted was bigger than the accumulator then the next instruction was skipped. 
Bearing no resemblance to common instructions, this Reverse Subtract instruction would be inherently inefficient. This is because of the large number of instructions needed to perform any useful operation. 
 
My project was originally intended to be an implementation of a computer to execute this instruction.
However, I changed my plans on reading an article in the 
June 1988 edition of Computer Architecture News \cite{jones:urisc}.
This article described how a computer could be built using an instruction to move
the contents of one memory location to another address.
Active elements within the computer's memory provide
program control and mathematical functions.
This `MOVE' instruction was  far more intuitive, and more efficient  for a computer to support,  so I decided to design and build such a computer. 
It offered the following advantages over the older design:-
\begin{itemize}
\item the article provided a much clearer description of the computer
\item machine code programs should be easier to produce
\item the performance of such a computer should be faster due to the ability to
support operations other than subtraction.
 \item it is a  flexible architecture, permitting much further expansion.
\end{itemize}

{\samepage
During the year  research  revealed the existence of three previous implementations of a single instruction computer. Two were based on the van der Poel instruction, while the third was MOVE based ---even though it predated the article I used.
\begin{enumerate}

\item Negev University, 1976 \cite{tabak:risc}\\
		A MOVE based instruction with four index registers was
		implemented in an eight bit wide computer.

\item Manchester University, 1987\\
		A final year CS undergraduate built a computer based upon
		the reverse subtract instruction. 

\item Philips Research labs, 1987 \cite{slav:oisc}\\
		A team of people implemented van der Poel's  computer, 
		with a PL/0 interpreter to execute pascal code.
		 The performance of this was claimed to be significantly slower 
		 than a M68000 based system.

\end{enumerate}
}

A major factor separating my design from these is the fact that 
the architecture was specified formally. This provides:-
\begin{itemize}
\item
A specification for software developers to use.
\item
A  simulation of the computer
\item
The ability to verify the hardware against the specification.
\end{itemize}
A fellow student has produced a compiler for the 
Ultimate RISC. 
This means the final product  consists of a 
computer interfaced to a host, which can cross-compile Pascal programs 
 onto my computer.  The design process therefore included 
 consultations with this other student, to enable 
the machine and the compiler to work together. 
While more akin to 
how an actual computer would be produced than a hardware project 
alone, it  meant  all design decisions were  the result of 
heated  discussion. 
